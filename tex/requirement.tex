\chapter{需求分析}
\label{requirement}

本章节将会对于系统本身的技术需求、编著和交互系统的目标用户的使用需求进行分析,以此作为系统设计和实现的指导。

\section{技术需求}
对于增强现实应用来说,为了向用户提供尽可能舒适的使用体验,需要满足一定的技术需求。\cite{artech}

\begin{itemize}
    \item \textbf{视场角(Field of View,FOV)}
    
       人的视场角是由人的视野决定的。人的双眼可以检测到动作的最远边界,大约是水平200度,竖直140度的范围,其中水平区域大约有140度是双目可见的范围,其余仅单目可见。用户使用手机进行交互,可以基本满足在双目可见的范围内成像。
    
    \item \textbf{像素度(Pixel Per Degree,PPD)}
    
     在理想的使用条件下,如果需要用户不能在观察电子屏幕成像物体的时候区分其中的像素,就需要显示器的分辨率可以满足在人眼成像的时候一定范围内的像素数目足够高。一般60PPD就可以达到要求。它与用户与显示屏的距离以及显示屏的分辨率有关。
    
    \item \textbf{延迟(latency)}
    
      人眼是具有一定的延迟的,因此对于应用的帧率具有一定的要求,帧率太低会影响使用体验,一般来说,帧率应当达到60帧。
    
    \item \textbf{视觉调节(accommodation)}
    
    人眼在观察深度不同的物体的时候人眼肌肉会进行自我调节,因此不同深度的物体在现实过程中不应过多,一般5-10个不同聚焦深度的物体就足够了,这样可以缓解人眼疲劳。
\end{itemize}

\indent    	由上述技术要求,在硬件具备一定的条件的基础上,用户与设备的距离应当适中。此外,软件层面,前端应用的渲染帧率应当达到60帧,并且同时渲染的虚拟物体不宜过多。

\section{用户需求}
本系统前端的编著与交互系统是面向教育者和学生开发的。因此在需求设计上,应当考虑他们的需求。这类用户需要编辑虚拟场景,但是本身没有编程能力,因此所有功能都应该封装起来,只留下用户接口。此外,为了降低用户的学习成本,应当在操作、指示符合直觉的基础上,适当的进行操作指示和辅助。最后,由于项目最终会移植到移动端,因此所有操作都应该可以兼容手机,例如鼠标悬停等操作就需要避免或进行修改。

本项目基于化学引擎进行开发,得出需要实现的功能包括,用户可以编辑实验环境,如场景灯光颜色和亮度、相机高度和角度等。用户可以添加实验仪器或药品,移动他们的位置、编辑药品的种类和量等。用户可以修改提示信息的颜色、位置、大小。用户可以自主编辑实验流程,包括实验的标题,每一个流程的名称(如预热),每个流程之下个步骤的名称(如点燃酒精灯)以及对应的事件名称等。事件用于在进行实验的过程中将用户自定义的实验步骤与统一的实验操作进行对应。
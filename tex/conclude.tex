\chapter{总结和展望}
\label{conclude}
本章节对项目进行总结,包括项目完成内容以及存在的缺陷,并且提出未来可能的发展方向。

\section{项目总结}
本项目实现了一个具有真实触感的增强现实应用。它具有服务器和客户端两部分。服务器通过物体追踪,实现对于预置模型的追踪。客户端可以通过编著系统,在PC端或安卓手机端创建新的实验,编辑实验内容、流程、场景等。之后,用户可以在安卓手机端与服务器互传数据,控制服务器,并且通过摄像机和二维码实现物体追踪与虚实融合,实现在增强现实应用中仍然具有真实触感。

本系统综合使用了多种技术和工具。服务器端使用C++开发,使用LibISR软件进行物体追踪,Libfreenect 2获取深度摄像头数据等。前端使用C\#开发,基于Unity引擎实现物体交互,Vuforia辅助实现虚实融合。两者还通过Protobuf、TCP/IP建立不同语言和系统的通信。

本系统基本实现了技术需求和用户需求。编著系统基本实现了一个用户创建一个实验的完整流程。使用增强现实应用可以基本实现虚实融合,并且提供了使用样例,而且物体的更新频率在服务器、网络传输、Unity应用三者都达到了可以接受的程度。

在使用工具的基础上,在服务器端撰写C++和CUDA代码,实现图像处理、网络传输、硬件驱动等功能。在客户端运用引擎提供的接口,并撰写C\#代码,创建了一套从管理器,到各个场景UI、引擎、持久层全覆盖的应用,实现了编著系统和虚实融合功能,并且支持多分辨率和多平台。

但是,目前追踪系统也存在一些问题。首先,单个物体的追踪效果并不是非常理想,特别是在环境光比较强的时候,深度判断的误差会增大。而且物体的移动范围、速度、角度都有一定程度的限制。其次,LibISR算法本身的限制,导致了在追踪结果发生错误之后,追踪很难回到正确的结果上,只能将系统追踪进行重置,这对于用户使用是比较麻烦的。

此外,LibISR的算法支持多个物体同时追踪,但是目前仅支持单个物体的追踪。因为在单个物体追踪效果比较差的情况下,多个物体将会使情况恶化。同时,多个物体需要对两个代价函数进行局部最小值求解,性能开销较大。

最后,目前仅支持少数几种物体模型的追踪,如果自定义物体,需要通过三维数据对物体进行3D打印生成,或者将既有模型进行三维扫描从而获得三维数据。此外,LibISR是通过二进制文件获取用户模型的,需要针对代码推断模型的编码方式。

目前编著系统也存在一些问题。首先对于用户的交互设计还不够完善。项目缺乏使用引导,一些UI的实际效果会与目标存在一定的偏差,会出现用户对于使用流程比较困惑的问题。例如,理想状态下希望用户在编辑实验时,首先添加物体,然后修改物体的位置和内容,之后编辑实验流程。但是实际使用后,发现用户很难在没有引导的情况下自主完成创建一个实验的完整流程。其次,用户的输入检查也不够充分,会导致在移动物体的时候出现叠加、物体添加物太多、物体移动出实验区域等问题。之后,UI设计也存在不合理之处,不同的按钮之间区分不够明显,用户可能会找不到正确的按钮。最后,项目目前支持的器具和药品种类仍然比较有限,而且没有给用户提供自主添加的接口。

同时,交互系统也存在一些缺陷。目前使用手机端进行增强现实显示,因此用户需要一只手拿手机,另一只手移动物体,操作不够流畅。此外,在每次使用前都要根据使用环境进行标定,这一步虽然现在可以在手机端直接完成,但是仍然是一个不便利的因素。

\section{项目展望}
首先项目可以对于现在存在的缺陷进行修改和补足,例如,物体追踪系统可以添加更多的模型,同时追踪更多数目的物体等,并且简化,或自动化标定流程。另外,编著系统可以在系统中添加引导程序,并且根据用户的实际使用情况来添加输入检查,并且更新UI设计。还可以在编著系统中添加更多的药品和工具的支持。此外,可以用头戴式显示屏或其他形式的屏幕代替手机,从而获得更好的交互体验。

目前编著系统和交互系统还不能很好地融合起来,例如用户进行物体追踪的交互场景并不能完全反映出在编著系统中编辑好的实验。

其次,项目本身仍然有很多可以充实的部分。物体追踪仍然是目前学界在不断研究的一个课题,不断有新的方法和工具被提出,可以作为提升物体追踪效果的指导。另外,对于编著系统而言,目前仍然在虚拟场景中进行编辑,未来可以移植到增强现实或虚拟现实场景中,为用户编著提供更加真实的交互体验。此外,也可以实现更多的用户编著操作,例如添加实验辅助要求、添加错误操作警示实验等,从而将该系统扩展到更多的应用场景中。

最后,可以以本项目为起点,开发更多的应用。目前应用是基于化学实验进行实现的,可以利用追踪系统的服务器,以及编著系统的框架,实现更多学科模拟实验,如物理、生物等。此外,这样一套基于物体追踪的增强现实应用在其他领域,如工厂装配、医学、产品展示等领域都可以有一定的应用。